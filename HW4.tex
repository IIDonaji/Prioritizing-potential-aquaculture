% Options for packages loaded elsewhere
% Options for packages loaded elsewhere
\PassOptionsToPackage{unicode}{hyperref}
\PassOptionsToPackage{hyphens}{url}
\PassOptionsToPackage{dvipsnames,svgnames,x11names}{xcolor}
%
\documentclass[
  landscapepaper,
  DIV=11,
  numbers=noendperiod]{scrartcl}
\usepackage{xcolor}
\usepackage[margin= 0.5in]{geometry}
\usepackage{amsmath,amssymb}
\setcounter{secnumdepth}{5}
\usepackage{iftex}
\ifPDFTeX
  \usepackage[T1]{fontenc}
  \usepackage[utf8]{inputenc}
  \usepackage{textcomp} % provide euro and other symbols
\else % if luatex or xetex
  \usepackage{unicode-math} % this also loads fontspec
  \defaultfontfeatures{Scale=MatchLowercase}
  \defaultfontfeatures[\rmfamily]{Ligatures=TeX,Scale=1}
\fi
\usepackage{lmodern}
\ifPDFTeX\else
  % xetex/luatex font selection
\fi
% Use upquote if available, for straight quotes in verbatim environments
\IfFileExists{upquote.sty}{\usepackage{upquote}}{}
\IfFileExists{microtype.sty}{% use microtype if available
  \usepackage[]{microtype}
  \UseMicrotypeSet[protrusion]{basicmath} % disable protrusion for tt fonts
}{}
\makeatletter
\@ifundefined{KOMAClassName}{% if non-KOMA class
  \IfFileExists{parskip.sty}{%
    \usepackage{parskip}
  }{% else
    \setlength{\parindent}{0pt}
    \setlength{\parskip}{6pt plus 2pt minus 1pt}}
}{% if KOMA class
  \KOMAoptions{parskip=half}}
\makeatother
% Make \paragraph and \subparagraph free-standing
\makeatletter
\ifx\paragraph\undefined\else
  \let\oldparagraph\paragraph
  \renewcommand{\paragraph}{
    \@ifstar
      \xxxParagraphStar
      \xxxParagraphNoStar
  }
  \newcommand{\xxxParagraphStar}[1]{\oldparagraph*{#1}\mbox{}}
  \newcommand{\xxxParagraphNoStar}[1]{\oldparagraph{#1}\mbox{}}
\fi
\ifx\subparagraph\undefined\else
  \let\oldsubparagraph\subparagraph
  \renewcommand{\subparagraph}{
    \@ifstar
      \xxxSubParagraphStar
      \xxxSubParagraphNoStar
  }
  \newcommand{\xxxSubParagraphStar}[1]{\oldsubparagraph*{#1}\mbox{}}
  \newcommand{\xxxSubParagraphNoStar}[1]{\oldsubparagraph{#1}\mbox{}}
\fi
\makeatother

\usepackage{color}
\usepackage{fancyvrb}
\newcommand{\VerbBar}{|}
\newcommand{\VERB}{\Verb[commandchars=\\\{\}]}
\DefineVerbatimEnvironment{Highlighting}{Verbatim}{commandchars=\\\{\}}
% Add ',fontsize=\small' for more characters per line
\usepackage{framed}
\definecolor{shadecolor}{RGB}{241,243,245}
\newenvironment{Shaded}{\begin{snugshade}}{\end{snugshade}}
\newcommand{\AlertTok}[1]{\textcolor[rgb]{0.68,0.00,0.00}{#1}}
\newcommand{\AnnotationTok}[1]{\textcolor[rgb]{0.37,0.37,0.37}{#1}}
\newcommand{\AttributeTok}[1]{\textcolor[rgb]{0.40,0.45,0.13}{#1}}
\newcommand{\BaseNTok}[1]{\textcolor[rgb]{0.68,0.00,0.00}{#1}}
\newcommand{\BuiltInTok}[1]{\textcolor[rgb]{0.00,0.23,0.31}{#1}}
\newcommand{\CharTok}[1]{\textcolor[rgb]{0.13,0.47,0.30}{#1}}
\newcommand{\CommentTok}[1]{\textcolor[rgb]{0.37,0.37,0.37}{#1}}
\newcommand{\CommentVarTok}[1]{\textcolor[rgb]{0.37,0.37,0.37}{\textit{#1}}}
\newcommand{\ConstantTok}[1]{\textcolor[rgb]{0.56,0.35,0.01}{#1}}
\newcommand{\ControlFlowTok}[1]{\textcolor[rgb]{0.00,0.23,0.31}{\textbf{#1}}}
\newcommand{\DataTypeTok}[1]{\textcolor[rgb]{0.68,0.00,0.00}{#1}}
\newcommand{\DecValTok}[1]{\textcolor[rgb]{0.68,0.00,0.00}{#1}}
\newcommand{\DocumentationTok}[1]{\textcolor[rgb]{0.37,0.37,0.37}{\textit{#1}}}
\newcommand{\ErrorTok}[1]{\textcolor[rgb]{0.68,0.00,0.00}{#1}}
\newcommand{\ExtensionTok}[1]{\textcolor[rgb]{0.00,0.23,0.31}{#1}}
\newcommand{\FloatTok}[1]{\textcolor[rgb]{0.68,0.00,0.00}{#1}}
\newcommand{\FunctionTok}[1]{\textcolor[rgb]{0.28,0.35,0.67}{#1}}
\newcommand{\ImportTok}[1]{\textcolor[rgb]{0.00,0.46,0.62}{#1}}
\newcommand{\InformationTok}[1]{\textcolor[rgb]{0.37,0.37,0.37}{#1}}
\newcommand{\KeywordTok}[1]{\textcolor[rgb]{0.00,0.23,0.31}{\textbf{#1}}}
\newcommand{\NormalTok}[1]{\textcolor[rgb]{0.00,0.23,0.31}{#1}}
\newcommand{\OperatorTok}[1]{\textcolor[rgb]{0.37,0.37,0.37}{#1}}
\newcommand{\OtherTok}[1]{\textcolor[rgb]{0.00,0.23,0.31}{#1}}
\newcommand{\PreprocessorTok}[1]{\textcolor[rgb]{0.68,0.00,0.00}{#1}}
\newcommand{\RegionMarkerTok}[1]{\textcolor[rgb]{0.00,0.23,0.31}{#1}}
\newcommand{\SpecialCharTok}[1]{\textcolor[rgb]{0.37,0.37,0.37}{#1}}
\newcommand{\SpecialStringTok}[1]{\textcolor[rgb]{0.13,0.47,0.30}{#1}}
\newcommand{\StringTok}[1]{\textcolor[rgb]{0.13,0.47,0.30}{#1}}
\newcommand{\VariableTok}[1]{\textcolor[rgb]{0.07,0.07,0.07}{#1}}
\newcommand{\VerbatimStringTok}[1]{\textcolor[rgb]{0.13,0.47,0.30}{#1}}
\newcommand{\WarningTok}[1]{\textcolor[rgb]{0.37,0.37,0.37}{\textit{#1}}}

\usepackage{longtable,booktabs,array}
\usepackage{calc} % for calculating minipage widths
% Correct order of tables after \paragraph or \subparagraph
\usepackage{etoolbox}
\makeatletter
\patchcmd\longtable{\par}{\if@noskipsec\mbox{}\fi\par}{}{}
\makeatother
% Allow footnotes in longtable head/foot
\IfFileExists{footnotehyper.sty}{\usepackage{footnotehyper}}{\usepackage{footnote}}
\makesavenoteenv{longtable}
\usepackage{graphicx}
\makeatletter
\newsavebox\pandoc@box
\newcommand*\pandocbounded[1]{% scales image to fit in text height/width
  \sbox\pandoc@box{#1}%
  \Gscale@div\@tempa{\textheight}{\dimexpr\ht\pandoc@box+\dp\pandoc@box\relax}%
  \Gscale@div\@tempb{\linewidth}{\wd\pandoc@box}%
  \ifdim\@tempb\p@<\@tempa\p@\let\@tempa\@tempb\fi% select the smaller of both
  \ifdim\@tempa\p@<\p@\scalebox{\@tempa}{\usebox\pandoc@box}%
  \else\usebox{\pandoc@box}%
  \fi%
}
% Set default figure placement to htbp
\def\fps@figure{htbp}
\makeatother





\setlength{\emergencystretch}{3em} % prevent overfull lines

\providecommand{\tightlist}{%
  \setlength{\itemsep}{0pt}\setlength{\parskip}{0pt}}



 


\usepackage{booktabs}
\usepackage{longtable}
\usepackage{array}
\usepackage{multirow}
\usepackage{wrapfig}
\usepackage{float}
\usepackage{colortbl}
\usepackage{pdflscape}
\usepackage{tabu}
\usepackage{threeparttable}
\usepackage{threeparttablex}
\usepackage[normalem]{ulem}
\usepackage{makecell}
\usepackage{xcolor}
\KOMAoption{captions}{tableheading}
\makeatletter
\@ifpackageloaded{caption}{}{\usepackage{caption}}
\AtBeginDocument{%
\ifdefined\contentsname
  \renewcommand*\contentsname{Table of contents}
\else
  \newcommand\contentsname{Table of contents}
\fi
\ifdefined\listfigurename
  \renewcommand*\listfigurename{List of Figures}
\else
  \newcommand\listfigurename{List of Figures}
\fi
\ifdefined\listtablename
  \renewcommand*\listtablename{List of Tables}
\else
  \newcommand\listtablename{List of Tables}
\fi
\ifdefined\figurename
  \renewcommand*\figurename{Figure}
\else
  \newcommand\figurename{Figure}
\fi
\ifdefined\tablename
  \renewcommand*\tablename{Table}
\else
  \newcommand\tablename{Table}
\fi
}
\@ifpackageloaded{float}{}{\usepackage{float}}
\floatstyle{ruled}
\@ifundefined{c@chapter}{\newfloat{codelisting}{h}{lop}}{\newfloat{codelisting}{h}{lop}[chapter]}
\floatname{codelisting}{Listing}
\newcommand*\listoflistings{\listof{codelisting}{List of Listings}}
\makeatother
\makeatletter
\makeatother
\makeatletter
\@ifpackageloaded{caption}{}{\usepackage{caption}}
\@ifpackageloaded{subcaption}{}{\usepackage{subcaption}}
\makeatother
\usepackage{bookmark}
\IfFileExists{xurl.sty}{\usepackage{xurl}}{} % add URL line breaks if available
\urlstyle{same}
\hypersetup{
  pdftitle={Prioritizing-potential-aquaculture},
  pdfauthor={Ixel},
  colorlinks=true,
  linkcolor={blue},
  filecolor={Maroon},
  citecolor={Blue},
  urlcolor={Blue},
  pdfcreator={LaTeX via pandoc}}


\title{Prioritizing-potential-aquaculture}
\author{Ixel}
\date{2025-12-31}
\begin{document}
\maketitle

\renewcommand*\contentsname{Table of contents}
{
\hypersetup{linkcolor=}
\setcounter{tocdepth}{3}
\tableofcontents
}

\pandocbounded{\includegraphics[keepaspectratio]{fREADME1.png.png}}
\pandocbounded{\includegraphics[keepaspectratio]{fREADME1.2.png.png}}

What Exclusive Economic Zones (eez) on the west Coast of the US are best
suited to developing marine aquaculture for several species of oysters
and California datemussel.

\section{Set-Up}\label{set-up}

\begin{Shaded}
\begin{Highlighting}[]
\CommentTok{\# Load in Libraries}
\FunctionTok{library}\NormalTok{(here)}
\FunctionTok{library}\NormalTok{(testthat)}
\FunctionTok{library}\NormalTok{(tidyverse)}
\FunctionTok{library}\NormalTok{(sf)}
\FunctionTok{library}\NormalTok{(kableExtra)}
\FunctionTok{library}\NormalTok{(sp)}
\FunctionTok{library}\NormalTok{(stars) }
\FunctionTok{library}\NormalTok{(tmap)}
\FunctionTok{library}\NormalTok{(terra)}
\FunctionTok{library}\NormalTok{(viridisLite)}
\end{Highlighting}
\end{Shaded}

\begin{Shaded}
\begin{Highlighting}[]
\CommentTok{\# Prepare Data}

\CommentTok{\# Bathymetry}
\NormalTok{depth }\OtherTok{\textless{}{-}} \FunctionTok{read\_stars}\NormalTok{(}\FunctionTok{here}\NormalTok{(}\StringTok{"data/depth.tif"}\NormalTok{))}

\CommentTok{\# Exclusive Economic Zones (eez)}
\NormalTok{eez }\OtherTok{\textless{}{-}} \FunctionTok{st\_read}\NormalTok{(}\FunctionTok{here}\NormalTok{(}\StringTok{"data/wc\_regions\_clean.shp"}\NormalTok{))}

\CommentTok{\# Sea Surface Temperature (SST) load as a list of .tiff files }
\NormalTok{sst }\OtherTok{\textless{}{-}} \FunctionTok{list.files}\NormalTok{(}\AttributeTok{path=} \StringTok{"data"}\NormalTok{,}
                  \AttributeTok{pattern=} \StringTok{"average\_annual"}\NormalTok{, }\CommentTok{\# file name matching pattern}
                   \AttributeTok{full.names =} \ConstantTok{TRUE}\NormalTok{) }\CommentTok{\# Reference entire file name matching this pattern}

\CommentTok{\# Stack all rasters files in folder using rast()}
\NormalTok{sst }\OtherTok{\textless{}{-}} \FunctionTok{rast}\NormalTok{(sst)}
\FunctionTok{names}\NormalTok{(sst) }\CommentTok{\#checking list}
\end{Highlighting}
\end{Shaded}

Check that the datasets have the same coordinate reference system

\begin{Shaded}
\begin{Highlighting}[]
\CommentTok{\# Check and align CRS for multiple datasets}
\ControlFlowTok{if}\NormalTok{ (}\FunctionTok{st\_crs}\NormalTok{(depth) }\SpecialCharTok{==} \FunctionTok{st\_crs}\NormalTok{(eez)) \{}
  \FunctionTok{print}\NormalTok{(}\StringTok{"Coordinate reference system match!"}\NormalTok{)}
  
\NormalTok{\} }\ControlFlowTok{else}\NormalTok{\{}
  \FunctionTok{warning}\NormalTok{(}\StringTok{"CRS mismatch between depth and eez"}\NormalTok{)}
  \CommentTok{\# tranfgotm data to match}
\NormalTok{  depth }\OtherTok{\textless{}{-}} \FunctionTok{st\_transform}\NormalTok{(depth, }\FunctionTok{st\_crs}\NormalTok{(eez))}

\NormalTok{  \}}
\end{Highlighting}
\end{Shaded}

\begin{verbatim}
[1] "Coordinate reference system match!"
\end{verbatim}

\begin{Shaded}
\begin{Highlighting}[]
\ControlFlowTok{if}\NormalTok{ (}\FunctionTok{st\_crs}\NormalTok{(eez)}\SpecialCharTok{$}\NormalTok{wkt }\SpecialCharTok{!=} \FunctionTok{crs}\NormalTok{(sst)) \{}
  \FunctionTok{warning}\NormalTok{(}\StringTok{"CRS mismatch between eez and sst"}\NormalTok{)}
\NormalTok{  sst }\OtherTok{\textless{}{-}} \FunctionTok{project}\NormalTok{(sst, }\FunctionTok{st\_crs}\NormalTok{(eez)}\SpecialCharTok{$}\NormalTok{wkt)  }
\NormalTok{\}}
\end{Highlighting}
\end{Shaded}

\begin{verbatim}
Warning: CRS mismatch between eez and sst
\end{verbatim}

\section{Process data}\label{process-data}

\begin{itemize}
\tightlist
\item
  Find the mean SST from 2008-2012 (e.g.~create single raster of average
  SST)

  \begin{itemize}
  \tightlist
  \item
    convert average SST from Kelvin to Celsius hint: subtract by 273.15
  \end{itemize}
\item
  crop depth raster to match the extent of the SST raster note:

  \begin{itemize}
  \tightlist
  \item
    the resolutions of the SST and depth data do not match resample the
    depth data to match the resolution of the SST data using the nearest
    neighbor approach
  \item
    check that the depth and SST match in resolution, extent, and
    coordinate reference system
  \end{itemize}
\end{itemize}

\begin{Shaded}
\begin{Highlighting}[]
\CommentTok{\# Calculate the SST average }
\NormalTok{mean\_sst }\OtherTok{\textless{}{-}} \FunctionTok{app}\NormalTok{(sst[[}\FunctionTok{c}\NormalTok{(}\StringTok{"average\_annual\_sst\_2008"}\NormalTok{,}
                       \StringTok{"average\_annual\_sst\_2009"}\NormalTok{, }
                       \StringTok{"average\_annual\_sst\_2010"}\NormalTok{,}
                       \StringTok{"average\_annual\_sst\_2011"}\NormalTok{, }
                       \StringTok{"average\_annual\_sst\_2012"}\NormalTok{)]], }
                \AttributeTok{fun =}\NormalTok{ mean, }\AttributeTok{na.rm =} \ConstantTok{TRUE}\NormalTok{)}

\CommentTok{\# Convert average SST from Kelvin to Celsius}
\NormalTok{sst\_mean\_celsius }\OtherTok{\textless{}{-}}\NormalTok{ mean\_sst }\SpecialCharTok{{-}} \FloatTok{273.15}
\end{Highlighting}
\end{Shaded}

\begin{Shaded}
\begin{Highlighting}[]
\CommentTok{\# First convert depth stars to SpatRaster}
\NormalTok{depth\_rast }\OtherTok{\textless{}{-}} \FunctionTok{rast}\NormalTok{(depth)}
\CommentTok{\# crop depth to match SST  extent}
\NormalTok{depth\_crop }\OtherTok{\textless{}{-}} \FunctionTok{crop}\NormalTok{(depth\_rast, sst\_mean\_celsius)}

\CommentTok{\# Resample depth\_crop to match SST resolution, use method near}
\NormalTok{depth\_resample }\OtherTok{\textless{}{-}} \FunctionTok{resample}\NormalTok{(depth\_crop,}
\NormalTok{                           sst\_mean\_celsius, }
                           \AttributeTok{method =} \StringTok{"near"}\NormalTok{)}
\end{Highlighting}
\end{Shaded}

\begin{Shaded}
\begin{Highlighting}[]
\CommentTok{\#Check resolution match }
\ControlFlowTok{if}\NormalTok{ (}\FunctionTok{all}\NormalTok{(}\FunctionTok{dim}\NormalTok{(depth\_resample) }\SpecialCharTok{==} \FunctionTok{dim}\NormalTok{(sst\_mean\_celsius)) }\SpecialCharTok{\&\&}
    \FunctionTok{isTRUE}\NormalTok{(}\FunctionTok{all.equal}\NormalTok{(}\FunctionTok{res}\NormalTok{(depth\_resample), }\FunctionTok{res}\NormalTok{(sst\_mean\_celsius))) }\SpecialCharTok{\&\&}
    \FunctionTok{crs}\NormalTok{(depth\_resample) }\SpecialCharTok{==} \FunctionTok{crs}\NormalTok{(sst\_mean\_celsius) }\SpecialCharTok{\&\&}
    \FunctionTok{ext}\NormalTok{(depth\_resample) }\SpecialCharTok{==} \FunctionTok{ext}\NormalTok{(sst\_mean\_celsius)) \{}
  \FunctionTok{cat}\NormalTok{(}\StringTok{"✓ All raster properties match!}\SpecialCharTok{\textbackslash{}n}\StringTok{"}\NormalTok{)}
\NormalTok{\} }\ControlFlowTok{else}\NormalTok{ \{}
  \FunctionTok{cat}\NormalTok{(}\StringTok{"✗ Some properties do not match}\SpecialCharTok{\textbackslash{}n}\StringTok{"}\NormalTok{)}
\NormalTok{\}}
\end{Highlighting}
\end{Shaded}

\begin{verbatim}
✗ Some properties do not match
\end{verbatim}

\begin{Shaded}
\begin{Highlighting}[]
\CommentTok{\# Match depth and SST CRS\textquotesingle{}s}
\FunctionTok{crs}\NormalTok{(depth\_resample) }\OtherTok{\textless{}{-}} \FunctionTok{crs}\NormalTok{(sst\_mean\_celsius)}

\CommentTok{\# Stack}
\NormalTok{ocean\_stack }\OtherTok{\textless{}{-}} \FunctionTok{c}\NormalTok{(sst\_mean\_celsius, depth\_resample)}
\FunctionTok{names}\NormalTok{(ocean\_stack) }\OtherTok{\textless{}{-}} \FunctionTok{c}\NormalTok{(}\StringTok{"sst"}\NormalTok{, }\StringTok{"depth"}\NormalTok{)}
\FunctionTok{cat}\NormalTok{(}\StringTok{"✓ Successfully stacked!}\SpecialCharTok{\textbackslash{}n}\StringTok{"}\NormalTok{)}
\end{Highlighting}
\end{Shaded}

\begin{verbatim}
✓ Successfully stacked!
\end{verbatim}

\section{Find suitable locations for
Oysters}\label{find-suitable-locations-for-oysters}

To find suitable locations for marine aquaculture, we'll need to find
locations that are suitable in terms of both SST and depth.
\textbf{Oyster} research has show that oysters need the following
conditions for optimal growth: - Sea surface Reclassify SST and depth
data into locations that are suitable for oysters hint: set suitable
values to 1 and unsuitable values to 0 find locations that satisfy both
SST and depth conditions

\begin{Shaded}
\begin{Highlighting}[]
\CommentTok{\# Reclassify SST (11{-}30°C suitable for oysters)}
\NormalTok{sst\_suitable }\OtherTok{\textless{}{-}} \FunctionTok{app}\NormalTok{(sst\_mean\_celsius, }\AttributeTok{fun =} \ControlFlowTok{function}\NormalTok{(x)\{}
  \FunctionTok{ifelse}\NormalTok{(x }\SpecialCharTok{\textgreater{}=} \DecValTok{11} \SpecialCharTok{\&}\NormalTok{ x }\SpecialCharTok{\textless{}=} \DecValTok{30}\NormalTok{,}\DecValTok{1}\NormalTok{,}\DecValTok{0}\NormalTok{) }\CommentTok{\# if temperate is between 11{-} 30 °C, return 1, else 0}
\NormalTok{\})}

\CommentTok{\# Reclassify Suitable depth range (0{-}70 meters for oysters)}
\NormalTok{depth\_suitable }\OtherTok{\textless{}{-}} \FunctionTok{app}\NormalTok{(depth\_resample, }\AttributeTok{fun =} \ControlFlowTok{function}\NormalTok{(x)\{}
  \FunctionTok{ifelse}\NormalTok{(x }\SpecialCharTok{\textgreater{}=} \SpecialCharTok{{-}}\DecValTok{70} \SpecialCharTok{\&}\NormalTok{ x }\SpecialCharTok{\textless{}=} \DecValTok{0}\NormalTok{,}\DecValTok{1}\NormalTok{,}\DecValTok{0}\NormalTok{) }\CommentTok{\# select depths from 0 to 70 meters below surface.}
\NormalTok{\})}

\CommentTok{\# Suitable locations for both conditions}
\CommentTok{\#Use lapp() to multiply the two suitability raster}

\NormalTok{suitable\_locations }\OtherTok{\textless{}{-}} \FunctionTok{lapp}\NormalTok{(}\FunctionTok{c}\NormalTok{(sst\_suitable, depth\_suitable), }
                          \AttributeTok{fun =} \ControlFlowTok{function}\NormalTok{(sst, depth) \{}
\NormalTok{                             sst }\SpecialCharTok{*}\NormalTok{ depth}
\NormalTok{                           \})}

\CommentTok{\#suitable\_locations \textless{}{-} sst\_suitable * depth\_suitable}
\end{Highlighting}
\end{Shaded}

\section{Determine the most suitable
EEZ}\label{determine-the-most-suitable-eez}

We want to determine the total suitable area within each EEZ in order to
rank zones by priority. To do so, we need to find the total area of
suitable locations within each EEZ.

Select suitable cells within West Coast EEZ's find area of grid cells
find the total suitable area within each EEZ hint: it might be helpful
to rasterize the EEZ data

\begin{Shaded}
\begin{Highlighting}[]
\CommentTok{\# Rasterize the EEZ data (creates zone IDs)}
\NormalTok{eez\_raster }\OtherTok{\textless{}{-}} \FunctionTok{rasterize}\NormalTok{(eez, suitable\_locations, }\AttributeTok{field =} \StringTok{"rgn"}\NormalTok{)}

\CommentTok{\# Mask suitable locations to EEZ boundaries}
\NormalTok{suitable\_eez }\OtherTok{\textless{}{-}} \FunctionTok{mask}\NormalTok{(suitable\_locations, eez) }\CommentTok{\# keep only raster cells that fall within the EEZ polygons}

\CommentTok{\#Calculate cell area with cellSize()}
\NormalTok{cell\_area\_km2 }\OtherTok{\textless{}{-}} \FunctionTok{cellSize}\NormalTok{(suitable\_locations, }\AttributeTok{unit =} \StringTok{"km"}\NormalTok{) }\CommentTok{\# calculate actual area of each raster cell}

\CommentTok{\# Calculate suitable area per cell}
\NormalTok{suitable\_area }\OtherTok{\textless{}{-}}\NormalTok{ suitable\_eez }\SpecialCharTok{*}\NormalTok{ cell\_area\_km2}

\CommentTok{\# Sum suitable are by EEZ zonal statistics }
\NormalTok{eez\_suitable\_area }\OtherTok{\textless{}{-}} \FunctionTok{zonal}\NormalTok{(suitable\_area, eez\_raster, }\AttributeTok{fun =} \StringTok{"sum"}\NormalTok{, }\AttributeTok{na.rm =} \ConstantTok{TRUE}\NormalTok{)}
\FunctionTok{colnames}\NormalTok{(eez\_suitable\_area) }\OtherTok{\textless{}{-}} \FunctionTok{c}\NormalTok{(}\StringTok{"eez\_id"}\NormalTok{, }\StringTok{"suitable\_area\_km2"}\NormalTok{)}
\end{Highlighting}
\end{Shaded}

\begin{Shaded}
\begin{Highlighting}[]
\CommentTok{\# join with eez data}
\NormalTok{eez}\SpecialCharTok{$}\NormalTok{suitable\_area\_km2 }\OtherTok{\textless{}{-}}\NormalTok{ eez\_suitable\_area}\SpecialCharTok{$}\NormalTok{suitable\_area\_km2[}\FunctionTok{match}\NormalTok{(eez}\SpecialCharTok{$}\NormalTok{rgn, eez\_suitable\_area}\SpecialCharTok{$}\NormalTok{eez\_id)]}

\CommentTok{\# Rank EEZ data}
\NormalTok{eez\_ranked }\OtherTok{\textless{}{-}}\NormalTok{ eez[}\FunctionTok{order}\NormalTok{(}\SpecialCharTok{{-}}\NormalTok{eez}\SpecialCharTok{$}\NormalTok{suitable\_area\_km2), ]}

\NormalTok{summary\_table }\OtherTok{\textless{}{-}} \FunctionTok{data.frame}\NormalTok{(}
  \AttributeTok{Rank =} \DecValTok{1}\SpecialCharTok{:}\FunctionTok{nrow}\NormalTok{(eez\_ranked),}
  \AttributeTok{EEZ\_Name =}\NormalTok{ eez\_ranked}\SpecialCharTok{$}\NormalTok{rgn\_key,}
  \AttributeTok{Suitable\_Area\_km2 =} \FunctionTok{round}\NormalTok{(eez\_ranked}\SpecialCharTok{$}\NormalTok{suitable\_area\_km2, }\DecValTok{2}\NormalTok{),}
  \AttributeTok{Percent\_of\_Total =} \FunctionTok{round}\NormalTok{(}\DecValTok{100} \SpecialCharTok{*}\NormalTok{ eez\_ranked}\SpecialCharTok{$}\NormalTok{suitable\_area\_km2 }\SpecialCharTok{/} 
                           \FunctionTok{sum}\NormalTok{(eez\_ranked}\SpecialCharTok{$}\NormalTok{suitable\_area\_km2), }\DecValTok{1}\NormalTok{)}
\NormalTok{)}


\NormalTok{eez\_table }\OtherTok{\textless{}{-}} \FunctionTok{kable}\NormalTok{(summary\_table,}
      \AttributeTok{caption =} \StringTok{"EEZ Rankings by Suitable Aquaculture Area"}\NormalTok{,}
      \AttributeTok{col.names =} \FunctionTok{c}\NormalTok{(}\StringTok{"Rank"}\NormalTok{, }\StringTok{"EEZ Region"}\NormalTok{, }\StringTok{"Suitable Area (km²)"}\NormalTok{, }\StringTok{"\% of Total"}\NormalTok{),}
      \AttributeTok{align =} \FunctionTok{c}\NormalTok{(}\StringTok{"c"}\NormalTok{, }\StringTok{"l"}\NormalTok{, }\StringTok{"r"}\NormalTok{, }\StringTok{"r"}\NormalTok{)) }\SpecialCharTok{\%\textgreater{}\%}
  \FunctionTok{kable\_styling}\NormalTok{(}\AttributeTok{bootstrap\_options =} \FunctionTok{c}\NormalTok{(}\StringTok{"striped"}\NormalTok{, }\StringTok{"hold\_position"}\NormalTok{),}
                \AttributeTok{full\_width =} \ConstantTok{FALSE}\NormalTok{) }\SpecialCharTok{\%\textgreater{}\%}
  \FunctionTok{row\_spec}\NormalTok{(}\DecValTok{0}\NormalTok{, }\AttributeTok{bold =} \ConstantTok{TRUE}\NormalTok{, }\AttributeTok{background =} \StringTok{"\#4CAF50"}\NormalTok{, }\AttributeTok{color =} \StringTok{"white"}\NormalTok{) }\SpecialCharTok{\%\textgreater{}\%}
  \FunctionTok{row\_spec}\NormalTok{(}\DecValTok{1}\NormalTok{, }\AttributeTok{bold =} \ConstantTok{TRUE}\NormalTok{, }\AttributeTok{background =} \StringTok{"\#E8F5E9"}\NormalTok{)  }\CommentTok{\# Highlight top rank}

\NormalTok{eez\_table}
\end{Highlighting}
\end{Shaded}

\begin{longtable}[t]{clrr}
\caption{\label{tab:unnamed-chunk-9}EEZ Rankings by Suitable Aquaculture Area}\\
\toprule
\cellcolor[HTML]{4CAF50}{\textcolor{white}{\textbf{Rank}}} & \cellcolor[HTML]{4CAF50}{\textcolor{white}{\textbf{EEZ Region}}} & \cellcolor[HTML]{4CAF50}{\textcolor{white}{\textbf{Suitable Area (km²)}}} & \cellcolor[HTML]{4CAF50}{\textcolor{white}{\textbf{\% of Total}}}\\
\midrule
\cellcolor[HTML]{E8F5E9}{\textbf{1}} & \cellcolor[HTML]{E8F5E9}{\textbf{CA-C}} & \cellcolor[HTML]{E8F5E9}{\textbf{4940.04}} & \cellcolor[HTML]{E8F5E9}{\textbf{34.1}}\\
2 & CA-S & 4221.39 & 29.1\\
3 & WA & 3313.16 & 22.8\\
4 & OR & 1578.97 & 10.9\\
5 & CA-N & 454.30 & 3.1\\
\bottomrule
\end{longtable}

\begin{Shaded}
\begin{Highlighting}[]
\CommentTok{\# Map with tmap basic}
\FunctionTok{tm\_basemap}\NormalTok{(}\StringTok{"Esri.OceanBasemap"}\NormalTok{, }\AttributeTok{alpha =} \FloatTok{0.7}\NormalTok{) }\SpecialCharTok{+}
\FunctionTok{tm\_shape}\NormalTok{(eez\_ranked) }\SpecialCharTok{+}
  \FunctionTok{tm\_polygons}\NormalTok{(}\StringTok{"suitable\_area\_km2"}\NormalTok{,}
              \AttributeTok{palette =} \StringTok{"{-}mako"}\NormalTok{,}
               \AttributeTok{style =} \StringTok{"cont"}\NormalTok{,}
              \AttributeTok{title =} \StringTok{"Suitable Area (km²)"}\NormalTok{,}
              \AttributeTok{border.col =} \StringTok{"black"}\NormalTok{,}
              \AttributeTok{border.lwd =} \FloatTok{0.5}\NormalTok{) }\SpecialCharTok{+}
  \FunctionTok{tm\_layout}\NormalTok{(}\AttributeTok{main.title =} \StringTok{"Suitable Oyster Aquaculuture Area by EEZ"}\NormalTok{,}
            \AttributeTok{main.title.size =} \FloatTok{1.2}\NormalTok{,}
            \AttributeTok{legend.outsides =} \ConstantTok{TRUE}\NormalTok{,}
            \AttributeTok{frame =} \ConstantTok{FALSE}\NormalTok{) }\SpecialCharTok{+} 
    \FunctionTok{tm\_text}\NormalTok{(}\StringTok{"rgn"}\NormalTok{, }\CommentTok{\# Label by region}
          \AttributeTok{size =}\NormalTok{ .}\DecValTok{8}\NormalTok{, }\CommentTok{\# Adjust size}
          \AttributeTok{col =} \StringTok{"white"}\NormalTok{, }\CommentTok{\# Adjust text color}
          \AttributeTok{fontface =} \StringTok{"bold"}\NormalTok{, }\CommentTok{\# Labels are bolded}
          \AttributeTok{xmod =} \SpecialCharTok{{-}}\NormalTok{.}\DecValTok{5}\NormalTok{) }\SpecialCharTok{+} \CommentTok{\# Adjust .5 from the left}
  \FunctionTok{tm\_compass}\NormalTok{(}\AttributeTok{type =} \StringTok{"4star"}\NormalTok{, }\AttributeTok{position =} \FunctionTok{c}\NormalTok{(}\StringTok{"right"}\NormalTok{, }\StringTok{"top"}\NormalTok{)) }\SpecialCharTok{+}
  \FunctionTok{tm\_scalebar}\NormalTok{(}\AttributeTok{position =} \FunctionTok{c}\NormalTok{(}\StringTok{"left"}\NormalTok{, }\StringTok{"bottom"}\NormalTok{)) }
\end{Highlighting}
\end{Shaded}

\pandocbounded{\includegraphics[keepaspectratio]{HW4_files/figure-pdf/unnamed-chunk-10-1.pdf}}

\section{Workflow for other species using a
function}\label{workflow-for-other-species-using-a-function}

\begin{Shaded}
\begin{Highlighting}[]
\NormalTok{find\_suitable\_aquaculture }\OtherTok{\textless{}{-}} \ControlFlowTok{function}\NormalTok{(sst\_raster, }
\NormalTok{                                      depth\_raster, }
\NormalTok{                                      eez\_sf, }
\NormalTok{                                      sst\_min, }
\NormalTok{                                      sst\_max, }
\NormalTok{                                      depth\_min, }
\NormalTok{                                      depth\_max,}
\NormalTok{                                      species\_name) \{}
  
\CommentTok{\# Reclassify SST}
\NormalTok{sst\_suitable }\OtherTok{\textless{}{-}} \FunctionTok{app}\NormalTok{(sst\_raster, }\AttributeTok{fun =} \ControlFlowTok{function}\NormalTok{(x) \{}
    \FunctionTok{ifelse}\NormalTok{(x }\SpecialCharTok{\textgreater{}=}\NormalTok{ sst\_min }\SpecialCharTok{\&}\NormalTok{ x }\SpecialCharTok{\textless{}=}\NormalTok{ sst\_max, }\DecValTok{1}\NormalTok{, }\DecValTok{0}\NormalTok{)}
\NormalTok{  \})}
  
\CommentTok{\# Reclassify depth}
\NormalTok{depth\_suitable }\OtherTok{\textless{}{-}} \FunctionTok{app}\NormalTok{(depth\_raster, }\AttributeTok{fun =} \ControlFlowTok{function}\NormalTok{(x) \{}
    \FunctionTok{ifelse}\NormalTok{(x }\SpecialCharTok{\textgreater{}=}\NormalTok{ depth\_min }\SpecialCharTok{\&}\NormalTok{ x }\SpecialCharTok{\textless{}=}\NormalTok{ depth\_max, }\DecValTok{1}\NormalTok{, }\DecValTok{0}\NormalTok{)}
\NormalTok{  \})}
  
\CommentTok{\# Find suitable locations (both conditions)}
\NormalTok{suitable\_locations }\OtherTok{\textless{}{-}}\NormalTok{ sst\_suitable }\SpecialCharTok{*}\NormalTok{ depth\_suitable}
\FunctionTok{names}\NormalTok{(suitable\_locations) }\OtherTok{\textless{}{-}}\NormalTok{ species\_name }

\CommentTok{\# Mask to EEZ}
\NormalTok{suitable\_in\_eez }\OtherTok{\textless{}{-}} \FunctionTok{mask}\NormalTok{(suitable\_locations, eez\_sf)}
  
\CommentTok{\# Rasterize EEZ}
\NormalTok{eez\_rasterize }\OtherTok{\textless{}{-}} \FunctionTok{rasterize}\NormalTok{(eez\_sf, suitable\_locations, }\AttributeTok{field =} \StringTok{"rgn"}\NormalTok{)}
  
\CommentTok{\# Calculate cell areas}
\NormalTok{cell\_area\_km2 }\OtherTok{\textless{}{-}} \FunctionTok{cellSize}\NormalTok{(suitable\_locations, }\AttributeTok{unit =} \StringTok{"km"}\NormalTok{)}
 

\CommentTok{\# Calculate suitable area per cell (commented out for now)}
\NormalTok{suitable\_area }\OtherTok{\textless{}{-}}\NormalTok{ suitable\_in\_eez }\SpecialCharTok{*}\NormalTok{ cell\_area\_km2}
  

\CommentTok{\# Sum by EEZ zone}
\NormalTok{eez\_suitable\_area }\OtherTok{\textless{}{-}} \FunctionTok{zonal}\NormalTok{(}
\NormalTok{  suitable\_area, }
\NormalTok{  eez\_rasterize, }
  \AttributeTok{fun =} \StringTok{"sum"}\NormalTok{, }
  \AttributeTok{na.rm =} \ConstantTok{TRUE}\NormalTok{)}

\FunctionTok{colnames}\NormalTok{(eez\_suitable\_area) }\OtherTok{\textless{}{-}} \FunctionTok{c}\NormalTok{(}\StringTok{"EEZ\_ID"}\NormalTok{, }\StringTok{"Suitable\_Area\_km2"}\NormalTok{)}
  
\CommentTok{\# Join with EEZ data and rank}
\NormalTok{eez\_sf}\SpecialCharTok{$}\NormalTok{suitable\_area\_km2 }\OtherTok{\textless{}{-}}
\NormalTok{  eez\_suitable\_area}\SpecialCharTok{$}\NormalTok{Suitable\_Area\_km2[}
    \FunctionTok{match}\NormalTok{(eez\_sf}\SpecialCharTok{$}\NormalTok{rgn, eez\_suitable\_area}\SpecialCharTok{$}\NormalTok{EEZ\_ID)]}
\NormalTok{  eez\_ranked }\OtherTok{\textless{}{-}}\NormalTok{ eez\_sf[}\FunctionTok{order}\NormalTok{(}\SpecialCharTok{{-}}\NormalTok{eez\_sf}\SpecialCharTok{$}\NormalTok{suitable\_area\_km2), ]}
  
\CommentTok{\# Create summary table}
\NormalTok{summary\_table }\OtherTok{\textless{}{-}} \FunctionTok{data.frame}\NormalTok{(}
   \AttributeTok{Rank =} \DecValTok{1}\SpecialCharTok{:}\FunctionTok{nrow}\NormalTok{(eez\_ranked),}
    \AttributeTok{EEZ =}\NormalTok{ eez\_ranked}\SpecialCharTok{$}\NormalTok{rgn\_key,}
    \AttributeTok{Suitable\_Area\_km2 =} \FunctionTok{round}\NormalTok{(eez\_ranked}\SpecialCharTok{$}\NormalTok{suitable\_area\_km2, }\DecValTok{2}\NormalTok{)}
\NormalTok{  )}
  
\CommentTok{\# Return results as a list}
\NormalTok{results }\OtherTok{\textless{}{-}} \FunctionTok{list}\NormalTok{(}
   \AttributeTok{species =}\NormalTok{ species\_name,}
  \AttributeTok{sst\_suitable =}\NormalTok{ sst\_suitable,}
   \AttributeTok{depth\_suitable =}\NormalTok{ depth\_suitable,}
   \AttributeTok{suitable\_locations =}\NormalTok{ suitable\_locations,}
   \AttributeTok{suitable\_in\_eez =}\NormalTok{ suitable\_in\_eez,}
   \AttributeTok{eez\_ranked =}\NormalTok{ eez\_ranked,}
    \AttributeTok{summary\_table =}\NormalTok{ summary\_table}
\NormalTok{  )}
  
 \FunctionTok{return}\NormalTok{(results)}
\NormalTok{\}}
\end{Highlighting}
\end{Shaded}

\section{Apply function to species of
choice}\label{apply-function-to-species-of-choice}

\subsection{Adula californiensis (Philippi,
1847)}\label{adula-californiensis-philippi-1847}

\begin{itemize}
\tightlist
\item
  Common Name: California datemussel
\item
  Low Fishing Vulnerability

  \begin{itemize}
  \tightlist
  \item
    Depth range: 0-20 meters
  \item
    Temperature: 9.1-18.6 C
  \end{itemize}
\item
  Source: SeaLifeBase
\end{itemize}

\begin{Shaded}
\begin{Highlighting}[]
\CommentTok{\# Mussels (different requirements)}
\NormalTok{mussel\_results }\OtherTok{\textless{}{-}} \FunctionTok{find\_suitable\_aquaculture}\NormalTok{(}
  \AttributeTok{sst\_raster =}\NormalTok{ sst\_mean\_celsius,}
  \AttributeTok{depth\_raster =}\NormalTok{ depth\_resample,}
  \AttributeTok{eez\_sf =}\NormalTok{ eez,}
  \AttributeTok{sst\_min =} \FloatTok{9.1}\NormalTok{,}
  \AttributeTok{sst\_max =} \FloatTok{18.6}\NormalTok{,}
  \AttributeTok{depth\_min =} \DecValTok{0}\NormalTok{,}
  \AttributeTok{depth\_max =} \DecValTok{20}\NormalTok{,}
  \AttributeTok{species\_name =} \StringTok{"California Datemussel"}
\NormalTok{)}

\CommentTok{\# Access results}
\FunctionTok{kable}\NormalTok{(mussel\_results}\SpecialCharTok{$}\NormalTok{summary\_table)}
\end{Highlighting}
\end{Shaded}

\begin{longtable}[]{@{}rlr@{}}
\toprule\noalign{}
Rank & EEZ & Suitable\_Area\_km2 \\
\midrule\noalign{}
\endhead
\bottomrule\noalign{}
\endlastfoot
1 & WA & 260.91 \\
2 & CA-S & 124.68 \\
3 & CA-C & 103.50 \\
4 & OR & 76.65 \\
5 & CA-N & 33.36 \\
\end{longtable}

\begin{Shaded}
\begin{Highlighting}[]
\CommentTok{\# Plot Mussel EEZ}
\CommentTok{\#| fig.width: 15}
\CommentTok{\#| fig.height: 8}

\FunctionTok{tm\_basemap}\NormalTok{(}\StringTok{"Esri.OceanBasemap"}\NormalTok{, }\AttributeTok{alpha =} \FloatTok{0.7}\NormalTok{) }\SpecialCharTok{+}
\CommentTok{\#tm\_shape(mussel\_results$suitable\_in\_eez) +}
  \FunctionTok{tm\_raster}\NormalTok{() }\SpecialCharTok{+}
  \FunctionTok{tm\_shape}\NormalTok{(mussel\_results}\SpecialCharTok{$}\NormalTok{eez\_ranked) }\SpecialCharTok{+}
  \FunctionTok{tm\_polygons}\NormalTok{(}\StringTok{"suitable\_area\_km2"}\NormalTok{,}
              \AttributeTok{palette =} \StringTok{"{-}mako"}\NormalTok{,}
              \AttributeTok{style =} \StringTok{"cont"}\NormalTok{,}
              \AttributeTok{title =} \StringTok{"Suitable Area (km²)"}\NormalTok{, }\AttributeTok{boarder.col =} \StringTok{"black"}\NormalTok{,}
              \AttributeTok{border.lwd =} \FloatTok{0.5}\NormalTok{,}
\NormalTok{              ) }\SpecialCharTok{+}
  \FunctionTok{tm\_layout}\NormalTok{(}\AttributeTok{main.title =}\StringTok{"Suitable Mussel Aquaculuture Area by EEZ"}\NormalTok{,}
            \AttributeTok{main.title.size =}\FloatTok{1.2}\NormalTok{,}\AttributeTok{legend.outsides =}\ConstantTok{TRUE}\NormalTok{,}\AttributeTok{frame =}\ConstantTok{FALSE}\NormalTok{) }\SpecialCharTok{+}
  \FunctionTok{tm\_text}\NormalTok{(}\StringTok{"rgn"}\NormalTok{, }\CommentTok{\# Label by region}
          \AttributeTok{size =}\NormalTok{ .}\DecValTok{8}\NormalTok{, }\CommentTok{\# Adjust size}
          \AttributeTok{col =} \StringTok{"white"}\NormalTok{, }\CommentTok{\# Adjust text color}
          \AttributeTok{fontface =} \StringTok{"bold"}\NormalTok{, }\CommentTok{\# Labels are bolded}
          \AttributeTok{xmod =} \SpecialCharTok{{-}}\NormalTok{.}\DecValTok{5}\NormalTok{) }\SpecialCharTok{+} \CommentTok{\# Adjust .5 from the left}
  \FunctionTok{tm\_compass}\NormalTok{(}\AttributeTok{type =} \StringTok{"4star"}\NormalTok{, }
           \AttributeTok{position =} \FunctionTok{c}\NormalTok{(}\StringTok{"right"}\NormalTok{, }\StringTok{"top"}\NormalTok{),}
           \AttributeTok{size =} \DecValTok{2}\NormalTok{) }\SpecialCharTok{+}
\FunctionTok{tm\_scale\_bar}\NormalTok{(}\AttributeTok{position =} \FunctionTok{c}\NormalTok{(}\StringTok{"left"}\NormalTok{, }\StringTok{"bottom"}\NormalTok{),}
             \AttributeTok{text.size =} \FloatTok{0.7}\NormalTok{) }
\end{Highlighting}
\end{Shaded}

\pandocbounded{\includegraphics[keepaspectratio]{HW4_files/figure-pdf/unnamed-chunk-13-1.pdf}}

\section{Reflections}\label{reflections}

Reflections must be clear, accurate, and demonstrate a deep
understanding of the analysis performed

This analysis identified suitable locations for oyster aquaculture along
the West Coast EEZs by integrating sea surface temperature (SST) and
bathymetric data. The workflow demonstrated how geospatial analysis can
support marine resource management decisions by combining environmental
constraints with jurisdictional boundaries. In the analysis we found
that 2.3\% (based on cellsize) of the West Coast EEZ area meets the
suitability criteria for oyster cultivation (11-30°C SST, 0-70m depth).
CA-C (Central California) ranked highest with 4940.04 km² of suitable
area, suggesting it as the priority zone for aquaculture development.

\section{Cite}\label{cite}

Hall, S. J., Delaporte, A., Phillips, M. J., Beveridge, M. \& O'Keefe,
M. Blue Frontiers: Managing the Environmental Costs of Aquaculture (The
WorldFish Center, Penang, Malaysia, 2011).

Gentry, R. R., Froehlich, H. E., Grimm, D., Kareiva, P., Parke, M.,
Rust, M., Gaines, S. D., \& Halpern, B. S. Mapping the global potential
for marine aquaculture. Nature Ecology \& Evolution, 1, 1317-1324
(2017).︎

GEBCO Compilation Group (2022) GEBCO\_2022 Grid
(doi:10.5285/e0f0bb80-ab44-2739-e053-6c86abc0289c)

Commercially Important Species Occurring in United States (contiguous
states). (2025). Sealifebase.ca.
https://www.sealifebase.ca/country/CountryChecklist.php?c\_code=840\&vhabitat=commercial

Oliver, R. (2025, November 25). Homework Assignment 4. Github.io.
https://eds-223-geospatial.github.io/assignments/HW4.html




\end{document}
